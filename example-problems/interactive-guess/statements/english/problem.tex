\textit{This is an interactive problem. You have to use a \texttt{flush}
operation right after printing each line. For example, in C++ you should use
the function \texttt{fflush(stdout)}, in Java --- \texttt{System.out.flush()},
in Pascal --- \texttt{flush(output)} and in Python ---
\texttt{sys.stdout.flush()}.}

In this problem jury has some number $x$, and you have to guess it. The number
$x$ is always an integer from $1$ and to $n$, where $n$ is given to you at the
beginning.


You can make queries to the testing system. Each query is a single integer from
$1$ to $n$. Flush output stream after printing each query. There are two
different responses the testing program can provide:

\begin{itemize} \item the string ``\texttt{<}'' (without quotes), if the jury's
    number is less than the integer in your query; \item the string
    ``\texttt{>=}'' (without quotes), if the jury's number is greater or equal
    to the integer in your query.  \end{itemize}

When your program guessed the number $x$, print string ``\texttt{! x}'', where
$x$ is the answer, and \textbf{terminate your program normally} immediately
after flushing the output stream.

Your program is allowed to make no more than $25$ queries (not including
printing the answer) to the testing system.

\InputFile

Use standard input to read the responses to the queries.

The first line contains an integer $n$ ($1 \le n \le 10^6$)~--- maximum
possible jury's number.

Following lines will contain responses to your queries~--- strings ``\t{<}'' or
``\t{>=}''. The $i$-th line is a response to your $i$-th query. When your
program will guess the number print ``\texttt{! x}'', where $x$ is the answer
and terminate your program.

The testing system will allow you to read the response on the query only after
your program print the query for the system and perform \texttt{flush}
operation.

\OutputFile

To make the queries your program must use standard output.

Your program must print the queries~--- integer numbers $x_i$ ($1 \le x_i \le
n$), one query per line (do not forget ``\textit{end of line}'' after each
$x_i$). After printing each line your program must perform operation
\texttt{flush}.

Each of the values $x_i$ mean the query to the testing system. The response to
the query will be given in the input file after you flush output. In case your
program guessed the number $x$, print string ``\texttt{! x}'', where $x$ --- is
the answer, and terminate your program.

\SAMPLES
